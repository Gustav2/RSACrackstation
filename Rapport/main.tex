\documentclass{article}

% Language setting
% Replace `english' with e.g. `spanish' to change the document language
\usepackage[danish]{babel}

% Set page size and margins
% Replace `letterpaper' with `a4paper' for UK/EU standard size
\usepackage[a4paper,top=2cm,bottom=2cm,left=3cm,right=3cm,marginparwidth=1.75cm]{geometry}

% Useful packages
\usepackage{amsmath}
\usepackage{graphicx}
\usepackage[colorlinks=true, allcolors=black]{hyperref}
\usepackage{tikz}
\usetikzlibrary{calc}
\usepackage{anyfontsize}
\usepackage{sectsty}
\usepackage{jslistings}
\definecolor{codegreen}{rgb}{0,0.6,0}
\definecolor{codegray}{rgb}{0.5,0.5,0.5}
\definecolor{codepurple}{rgb}{0.58,0,0.82}

\lstdefinestyle{JavaScript}{
    commentstyle=\color{codegreen},
    keywordstyle=\color{magenta},
    numberstyle=\tiny\color{codegray},
    stringstyle=\color{codepurple},
    basicstyle=\ttfamily\footnotesize,
    breakatwhitespace=false,         
    breaklines=true,                 
    captionpos=b,                    
    keepspaces=true,                 
    numbers=left,                    
    numbersep=5pt,                  
    showspaces=false,                
    showstringspaces=false,
    showtabs=false,                  
    tabsize=4
}

\lstset{style=javascript}


\usepackage{wrapfig}
\usepackage[utf8]{inputenc}
\usepackage[T1]{fontenc}

\begin{document}

    \begin{titlepage}
        \begin{center}
            \vspace*{1cm}

            \textbf{RSACrackstation}

            \vspace{0.5cm}
            Rapport over RSACrackstation projektet

            \vspace{1.5cm}

            \textbf{Daniel Nettelfield og Gustav Nybro}

            \vfill

            A thesis presented for the degree of\\
            Doctor of Philosophy

            \vspace{0.8cm}



            Teknisk Gymnasium Silkeborg\\
            3x\\
            31/10 - 2022

        \end{center}
    \end{titlepage}

    \newpage



    \tableofcontents


    \section{Resume}\label{sec:resume}


    \section{Introduktion}\label{sec:introduktion}

    Your introduction goes here! Simply start writing your document and use the Recompile button to view the 
    updated PDF preview. Examples of commonly used commands and features are listed below, to help you get started.

    Once you're familiar with the editor, you can find various project settings in the Overleaf menu, accessed 
    via the button in the very top left of the editor. To view tutorials, user guides, and further documentation, 
    please visit our \href{https://www.overleaf.com/learn}{help library}, or head to our plans page to 
    \href{https://www.overleaf.com/user/subscription/plans}{choose your plan}.


    \section{Problemformulering}\label{sec:problemformulering}

    \subsection{Beskrivelse}\label{subsec:beskrivelse}
    Hjemmesiden er en side der kan faktorisere primtal ved hjælp at af API call til en ekstern database.
    Derudover kan hjemmesiden bruges til at bryde krypteringsalgoritmen RSA, og dertil dekryptere indtastet tekst.
    Derudover kan hjemmesiden også bruges til at kryptere tekst med RSA. Hjemmesidens målgruppe er primært CTF
    spillere, cybersikkerheds entusiaster og andre der skal kryptere og dekryptere med RSA.

    \subsection{Krav}\label{subsec:krav}


    \section{Hvad er en API?}\label{sec:hvad-er-en-api?}


    \section{Frontend}\label{sec:frontend}
    \begin{lstlisting}[language=JavaScript,label={lst:kode}]
    // Shows the snackbar for 5 seconds
    function showSnackbar(message) {
        $("#failText").text(message);
        $("#snackbar").addClass("show-bar");
        setTimeout(function () {
            $("#snackbar").removeClass("show-bar");
        }, 5000);
    }
    \end{lstlisting}


    \section{Backend}\label{sec:backend}


    \section{API call}\label{sec:api-call}

    \bibliographystyle{alpha}
    \bibliography{sample}

\end{document}